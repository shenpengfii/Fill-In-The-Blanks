\documentclass[12pt]{article}

\input{lesson_paper_config.tex}  % 导入配置文件

\newcommand\titleofdoc{} % 文档标题
\newcommand\GroupName{} % 小组名

\usepackage{fancyhdr}
\pagestyle{fancy}
\fancyhf{}
\fancyhead[C]{\small 中国石油大学(华东)}

\setcounter{section}{0}


% 正文
%%%%%%%%%%%%%%%%%%%%%%%%%%%%%%%%%%%%%%%%%%%%%%%%%%%%%%%%%%%%%%%%%%%%%%%%%%%%%%%%%%%%%%%%%%%%%%%%%%%%%%%%%%%%%%%%%%%%%%%
\begin{document}
\begin{sloppypar}  % 两端对齐命令

 % \input{sections/cover.tex}  % 封面文件

 
    \begin{center}
        \erhao \sffamily 一款基于BiLSTM-CRF的英文选词填空机器作答应用设计

        \vspace{0.3cm}

        \xiaosihao \ttfamily 申\quad 鹏\quad 飞$^1$,作\quad 者$^2$

        \xiaowuhao (1.智能科学与技术2102 2115040110;2.作者详细单位,省市 邮编)
    \end{center}

    \xiaowuhao{
        \noindent \sffamily 摘要: \normalfont 在英文选词填空题目中,文章中的句子空缺通常具有多个选项。
        条件随机场的前向算法能够计算出填入各选项后的句子的“前向分数”,从而判断出最佳的选项。
        由百度研究院提出的双向长短时记忆-条件随机场模型是Transformer时代之前最成熟的序列标注模型之一,该网络通过结合双向长短时记忆网络和条件随机场,有较强的结构表示能力。
        本次报告将详细介绍基于该模型的选词填空机器答题应用的设计原理和表现效果。

        \noindent \sffamily 关键词:\normalfont 完形填空;机器答题;双向长短时记忆网络;条件随机场
       }

       \begin{center}
        \sihao Machine Answering For English Fill-In-The-Blank Word Selection Quizzes Based On BiLSTM-CRF Segmentation

        \vspace{0.3cm}

        \xiaosihao Shen pengfei$^1$,NAME Name-name$^2$

        \xiaowuhao (1. Artifical Intelligence Science and Engineering 2102 2115040110; 2. Department, City, City Zip Code, China)

    \end{center}
    \xiaowuhao{
        \noindent \textbf{Abstract: }In English Fill-in-the-Blank word selection quizzes, there are multiple options for a blank. 
        The forward algorithm of conditional random field can `predict' the option with the highest forward score, so as to choose the most likely answer.
        The bidirectional long short-term memory - conditional random field model was one of the most mature sequence annotation models before the Transformer era, which includes a conditional random field layer.
        This paper will introduce in detail the design and performance of the fill-in-the-blank word selection machine based on this model.

        \noindent \textbf{Keywords: }Fill-in-the-Blank; Machine Answering; Bidirectional Long Short-Term Memory; Conditional Random Field
       }

  % 标题及摘要文件

 \vspace{0.5cm}  % 分隔 0.5cm
 

 \section{引言}  % 第一个section的标题(一般为引言,此处不分栏,因为根据 word 模板来看,这一行似乎是单独的,但是内容又是要分栏的,所以把第一个section的标题和内容分开了)
 	\begin{multicols*}{2}  % 正文开始分栏
 	
    “词性标注”是自然语言处理的经典任务之一,AI能根据输入的句子推测出对应词性序列。
目前,主流的模型网络都结合了条件随机场(CRF)层。
许多学者和工程师基于CRF的前向算法提出了基于序列标注的文本生成应用,如RNN的诗歌生成应用等。
在生成式算法中,模型会将“前向分数”最大的句子作为最终的生成结果。

此外,关于英文选词填空机器答题的应用,一些学者已经提出了一些比较独特的方法,如来自Google团队的MaskGAN网络。
Transformer框架问世后,互联网上出现了大量的基于BERT的中英文选词填空应用,如Unmask等。
这些网络较BiLSTM-CRF各有优缺点。
\subsection{OCR的作用}
123123123213123123引言内容。引言作为论文的开场白,应以简短的篇幅介绍论文的写作背景和目的,以及相关领域内前人所做的工作和研究概况,说明本研究与前人工作的关系,目前研究的热点、存在的问题及作者工作的意义。1、开门见山,不绕圈子。避免大篇幅地讲述历史渊源和立题研究过程。2、言简意赅,突出重点。不应过多叙述同行熟知的及教科书中的常识性内容,确有必要提及他人的研究成果和基本原理时,只需以引用参考文献的形势标出即可。在引言中提示本文的工作和观点时,意思应明确,语言应简练。3、引言的内容不要与摘要雷同,也不是摘要的注释。4、引言要简短,最好不要分段论述,不要插图、列表和数学公式。  % 第一个section的内容,接上面的标题
    
    \input{sections/sec2.tex}  % 第二个section(标题加内容)
    
    \input{sections/sec3.tex}  % 

    % ...后续section
    
    
    % 参考文献
   %  \bibliographystyle{plain}  % 参考文献的格式
   \vspace{0.5cm}
   	\nocite{*}
    \bibliography{sections/refs.bib}  % 参考文献bib文件
    \end{multicols*}
 
 
\end{sloppypar}
\end{document}

%%%%%%%%%%%%%%%%%%%%%%%%%%%%%%%%%%%%%%%%%%%%%%%%%%%%%%%%%%%%%%%%%%%%%%%%%%%%%%%%%%%%%%%%%%%%%%%%%%%%%%%%%%%%%%%%%%%%%%%%%%%%%%%%%%%%%%%%%%%%%%%%%%%%%%