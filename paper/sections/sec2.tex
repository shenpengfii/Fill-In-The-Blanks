\section{量的书写规则}
正文内容。正文、图表中的变量都要用斜体字母,对于矢量和张量使用黑斜体,只有pH采用正体;使用新标准规定的符号\cite{simonyan2014very};量的符号为单个拉丁字母或希腊字母;不能把量符号作为纯数使用;不能把化学符号作为量符号使用,代表物质的符号表示成右下标,具体物质的符号及其状态等置于与主符号齐线的圆括号中。

注意区分量的下标字母的正斜体:凡量符号和代表变动性数字及坐标轴的字母作下标,采用斜体字母。

正文中引用参考文献的标注方法,在引用处对引用的文献,按它们在论著中出现的先后用阿拉伯数字连续排序,将序号置于方括号内,并视具体情况把序号作为上角标或作为语句的组成部分。
\subsection{单位的书写规则}
正文内容。单位符号无例外的采用正体字母。注意区分单位符号的大小写:一般单位符号为小写体,来源于人名的单位符号首字母大写。体积单位升的符号为大写L。
\subsubsection{表格的规范化}
正文内容。表格的设计应该科学、明确、简洁,具有自明性。表格应采用三线表,项目栏不宜过繁,小表宽度小于7.5 cm,大表宽度为12~15cm 。表必须有中英文表序、表题。表中顶线与栏目线之间的部分叫项目栏,底线与栏目线之间的部分叫表身。表身中数字一般不带单位,百分数也不带百分号,应把单位符号和百分号等归并在栏目中。如果表中栏目中单位均相同,则可把共同的单位提出来标示在表格顶线上方的右端(不加“单位”二字)。表身中同一栏各行的数值应以个位(或小数点),且有效位数相同。上下左右相邻栏内的文字或数字相同时,应重复写出。

\begin{table}[H]
	\xiaowuhao
	\centering
	\caption{\label{tab11} \xiaowuhao \hei 表题}
	\begin{tabular}{llll}
		\toprule
		Model                 & SMO+DNN & PCA+DNN & DNN   \\
		\midrule
		Accuracy              & 0.994   & 0.938   & 0.914 \\
		Precision             & 0.995   & 0.934   & 0.891 \\
		Recall                & 0.995   & 0.918   & 0.882 \\
		\bottomrule
	\end{tabular}
\end{table}