“词性标注”是自然语言处理的经典任务之一,AI能根据输入的句子推测出对应词性序列。
目前,主流的模型网络都结合了条件随机场(CRF)层。
许多学者和工程师基于CRF的前向算法提出了基于序列标注的文本生成应用,如RNN的诗歌生成应用等。
在生成式算法中,模型会将“前向分数”最大的句子作为最终的生成结果。

此外,关于英文选词填空机器答题的应用,一些学者已经提出了一些比较独特的方法,如来自Google团队的MaskGAN网络。
Transformer框架问世后,互联网上出现了大量的基于BERT的中英文选词填空应用,如Unmask等。
这些网络较BiLSTM-CRF各有优缺点。
\subsection{OCR的作用}
123123123213123123引言内容。引言作为论文的开场白,应以简短的篇幅介绍论文的写作背景和目的,以及相关领域内前人所做的工作和研究概况,说明本研究与前人工作的关系,目前研究的热点、存在的问题及作者工作的意义。1、开门见山,不绕圈子。避免大篇幅地讲述历史渊源和立题研究过程。2、言简意赅,突出重点。不应过多叙述同行熟知的及教科书中的常识性内容,确有必要提及他人的研究成果和基本原理时,只需以引用参考文献的形势标出即可。在引言中提示本文的工作和观点时,意思应明确,语言应简练。3、引言的内容不要与摘要雷同,也不是摘要的注释。4、引言要简短,最好不要分段论述,不要插图、列表和数学公式。