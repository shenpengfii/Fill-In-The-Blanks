
    \begin{center}
        \erhao \sffamily 一款基于BiLSTM-CRF的英文选词填空机器作答应用设计

        \vspace{0.3cm}

        \xiaosihao \ttfamily 申\quad 鹏\quad 飞$^1$,作\quad 者$^2$

        \xiaowuhao (1.智能科学与技术2102 2115040110;2.作者详细单位,省市 邮编)
    \end{center}

    \xiaowuhao{
        \noindent \sffamily 摘要: \normalfont 在英文选词填空题目中,文章中的句子空缺通常具有多个选项。
        条件随机场的前向算法能够计算出填入各选项后的句子的“前向分数”,从而判断出最佳的选项。
        由百度研究院提出的双向长短时记忆-条件随机场模型是Transformer时代之前最成熟的序列标注模型之一,该网络通过结合双向长短时记忆网络和条件随机场,有较强的结构表示能力。
        本次报告将详细介绍基于该模型的选词填空机器答题应用的设计原理和表现效果。

        \noindent \sffamily 关键词:\normalfont 完形填空;机器答题;双向长短时记忆网络;条件随机场
       }

       \begin{center}
        \sihao Machine Answering For English Fill-In-The-Blank Word Selection Quizzes Based On BiLSTM-CRF Segmentation

        \vspace{0.3cm}

        \xiaosihao Shen pengfei$^1$,NAME Name-name$^2$

        \xiaowuhao (1. Artifical Intelligence Science and Engineering 2102 2115040110; 2. Department, City, City Zip Code, China)

    \end{center}
    \xiaowuhao{
        \noindent \textbf{Abstract: }In English Fill-in-the-Blank word selection quizzes, there are multiple options for a blank. 
        The forward algorithm of conditional random field can `predict' the option with the highest forward score, so as to choose the most likely answer.
        The bidirectional long short-term memory - conditional random field model was one of the most mature sequence annotation models before the Transformer era, which includes a conditional random field layer.
        This paper will introduce in detail the design and performance of the fill-in-the-blank word selection machine based on this model.

        \noindent \textbf{Keywords: }Fill-in-the-Blank; Machine Answering; Bidirectional Long Short-Term Memory; Conditional Random Field
       }

